\documentclass[journal]{IEEEtran}
%
% General
%
\usepackage[english]{babel}
\usepackage[utf8]{inputenc}
\usepackage[T1]{fontenc}

%
% Formatting
%
\usepackage{flushend}
\usepackage[hidelinks]{hyperref}
\usepackage{lettrine}
\usepackage{microtype}
\usepackage{pifont}

\usepackage{xcolor}
\definecolor{liu}{rgb}{0,0.7,0.9}

%
% Shortcuts
%
\newcommand{\linkmarker}{\textsuperscript{$\color{liu}\omega$}}
\newcommand{\link}[3][]{\href{#2}{#3#1\linkmarker}}

\newcommand\footfloat[1]{%
  \begingroup
  \renewcommand\thefootnote{}\footnote{#1}%
  \addtocounter{footnote}{-1}%
  \endgroup
}

\newcommand{\slab}[1]{\label{section:#1}}
\newcommand{\sref}[1]{Section~\ref{section:#1}}

\renewcommand{\date}[1]{{\bf #1} \enspace}
\newcommand{\sep}{\ding{72}~}


\title{Curriculum Vitae of Ivan Ukhov}
\author{
  Ivan Ukhov\\
  E-mail: \link[,]{mailto:ivan.ukhov@gmail.com}{ivan.ukhov@gmail.com}
  blog: \link[,]{https://ivanukhov.com}{ivanukhov.com}\\
  GitHub: \link[,]{https://github.com/IvanUkhov}{IvanUkhov}
  LinkedIn: \link[,]{https://www.linkedin.com/in/IvanUkhov/}{IvanUkhov}
  Twitter: \link{https://twitter.com/IvanUkhov}{IvanUkhov}
}

\begin{document}

\maketitle

\begin{abstract}
In this work, we summarize Ivan Ukhov's career, qualifications, education, and
other relevant aspects in a form suitable for consumption by potential
employers. The motivation is to make it easier for the latter to quickly obtain
a sufficiently comprehensive understanding of Ivan's profile and subsequently
decide if and how to proceed with the recruitment process.
\end{abstract}

\begin{IEEEkeywords}
  Data science,
  machine learning,
  mathematics,
  optimization,
  probability theory,
  software engineering,
  statistics.
\end{IEEEkeywords}

\bstctlcite{IEEEexample:BSTcontrol}

\section{Introduction}

\lettrine[findent=0.4em, nindent=0em]{\textbf{T}}{alent acquisition} becomes
increasingly more difficult, as there are many candidates with diverse skills.
Under these circumstances, it is important to be able to make a reasonably
accurate estimation if any given one would be a good fit for the job at hand.
Ivan Ukhov is one such candidate, and this paper presents his profile so as to
streamline the aforementioned estimation process.\footfloat{\linkmarker The
omega subscript indicates the presence of interactive content. It is no
accessible in print form; however, it is considered optional.}

Ivan's professional interest is in decision-making based on collected evidence
combined with prior domain knowledge, answering business questions via
inferential and predictive modeling, optimization of business processes by means
of learning from data, and development of data products. In this setting, he is
interested in such disciplines as mathematics, probability theory, statistics,
and machine learning.

\section{Work Experience}
When it comes to the work experience, there are have been two employers:
first LeoVegas and then Voi.

{\bf February 2022--Present}, Staff Machine Learning Engineer at Voi
\begin{itemize}
\item She sells seashells by the seashore.
\end{itemize}

{\bf November 2019--February 2022}, Head of Data Science at LeoVegas.
\begin{itemize}
\item \emph{In addition to} the initiatives listed below, led the team, ensuring
clarity and relevance of direction. Held group and individual meetings.
Performed planning and setting of common and individual objectives.
\end{itemize}

{\bf February 2018--February 2022}, Data Scientist at LeoVegas.
\begin{itemize}
\item Developed an experimentation platform for democratization and
decentralization of hypothesis testing at the company (Google Cloud Platform,
Bayesian statistics).

\item Developed a predictive model for identifying individuals who are at risk
of gambling addiction (gradient boosting, recurrent neural networks).

\item Developed an inferential model and a natural-language-processing model for
analyzing the results of customer surveys (Bayesian statistics, recurrent neural
networks).

\item Developed a data-driven artificial environment for optimizing promotion
campaigns via reinforcement learning (feedforward neural networks).

\item Developed an ecosystem for data-science projects, including pipelines for
data preparation, frameworks for predictive modeling, and infrastructures for
scheduling and serving of predictive models (Google Cloud Platform).

\item Participated in other projects led by the team, including causal inference
in observational studies, churn prediction, high-value prediction, inference of
spillover in marketing, lifetime-value prediction, marketing mix modeling, and
recommender systems.

\item Participated in the recruitment of data scientists. Mentored team members
in data science, machine learning, and software engineering. See also Teaching
Experience.

\item Led research projects on the topic of problem gambling in collaboration
with Stockholm University and Nottingham Trent University.
\end{itemize}

\section{Technical Skills}
Strong knowledge of Python and R, including their ecosystems. Strong knowledge
of SQL. Strong knowledge of TensorFlow. Strong knowledge of Google Cloud
Platform. Good knowledge of shell scripting, Go, and Rust. Good knowledge of
Docker and Apache Airflow. Good knowledge of software testing. Familiar with
C++, JavaScript, HTML, and CSS. Familiar with continuous integration and
deployment. Master of version control with Git. Master of Vim and LaTeX.

\section{Teaching Experience}
2018--2019 --- Supervised two Master's students at LeoVegas on the topic of
identifying problem gamblers using recurrent neural networks and on the topic
of managing promotion campaigns using reinforcement learning.

2011--2017 --- Supervised one Ph.D. student and five Master's students at
Linköping University in machine learning, software engineering, and computer
systems.

2011--2017 --- Assisted in the following undergraduate courses at Linköping
University: \link[,]{https://studieinfo.liu.se/kurs/tddb84/ht-2018}{Design
Patterns} \link[,]{https://www.ida.liu.se/~TDDD25/}{Distributed Systems}
\link[,]{https://www.ida.liu.se/~TDDI08/}{Embedded Systems Design}
\link[,]{https://www.ida.liu.se/~TDDD04/}{Software Testing} and
\link[.]{https://www.ida.liu.se/~TDTS07/}{System Design and Methodologies}

\section{Learning Experience}
2018--2019 --- Passed the following specializations on Coursera: Deep Learning,
Machine Learning with TensorFlow on Google Cloud Platform, and Data Engineering.

2011--2017 --- Passed the following graduate courses at Linköping University:
Advanced Data Models and Databases, Bayesian Learning, Data Mining and
Statistical Learning, Distributed Systems, Gaussian Random Processes, Malliavin
Calculus and Stochastic Integration, Natural Language Processing, Neural
Networks with Applications to Vision and Language, Probability Theory, Real-Time
and Embedded Systems, Stochastic Optimization, and Stochastic Processes.

\section{Academic Qualifications}
\emph{2017 --- Doctor of Philosophy in Computer Science}

Embedded Systems Laboratory, Department of Computer and Information Science,
Linköping University.

\emph{2010 --- Master of Science in Computer Science with honors}

Department of Information and Control Systems, Peter the Great Saint Petersburg
Polytechnic University.

\emph{2010 --- Specialist in Business Management with honors}

International Graduate School of Management, Peter the Great Saint Petersburg
Polytechnic University.

\emph{2008 --- Bachelor of Science in Computer Science with honors}

Department of Information and Control Systems, Peter the Great Saint Petersburg
Polytechnic University.

\section{Selected Scientific Publications}
\emph{2017 --- System-Level Analysis and Design under Uncertainty}
\cite{ukhov2017d}

Ph.D. thesis. Encompasses the individual research projects listed below.

\emph{2017 --- Fine-Grained Long-Range Prediction of Resource Usage in Computer
Clusters} \cite{ukhov2017b}

Technical report. Studied resource usage in a
\link[.]{https://github.com/google/cluster-data}{computer cluster} Constructed
an efficient pipeline for data processing and devised a recurrent neural network
for forecasting resource usage. Made extensive use of TensorFlow.

\emph{2017 --- Fast Synthesis of Power and Temperature Profiles for the
Development of Data-Driven Resource Managers} \cite{ukhov2017c}

Technical report. Studied network traffic in a
\link[.]{https://github.com/google/cluster-data}{computer cluster} Developed an
infrastructure for simulating systems processing user requests with the goal of
providing large amounts of synthetic but realistic data to facilitate the
development of resource managers powered by machine learning.

\emph{2017 --- Probabilistic Analysis of Electronic Systems via Adaptive
Hierarchical Interpolation} \cite{ukhov2017a}

Journal article. Developed an efficient framework for probabilistic analysis of
electronic systems based on adaptive hierarchical interpolation on sparse grids.
Leveraged advanced topics in numerical analysis.

\emph{2015 --- Temperature-Centric Reliability Analysis and Optimization of
Electronic Systems under Process Variation} \cite{ukhov2015}

Journal article. Developed an efficient probabilistic framework for reliability
analysis of electronic systems using polynomial regression and applied this
framework in the context of energy optimization.

\emph{2014 --- Probabilistic Analysis of Power and Temperature under Process
Variation for Electronic-System Design} \cite{ukhov2014b}

Journal article. Developed an efficient probabilistic framework for temperature
analysis of electronic systems based on polynomial regression. Made use of
advanced topics in probability theory and numerical analysis.

\emph{2014 --- Statistical Analysis of Process Variation Based on Indirect
Measurements for Electronic-System Design} \cite{ukhov2014a}

Conference paper. Developed an efficient statistical framework for
characterizing variations in parameters of a technological process based on
indirect measurements. Made extensive use of Bayesian inference.

\emph{2012 --- Steady-State Dynamic Temperature Analysis and Reliability
Optimization for Embedded Multiprocessor Systems} \cite{ukhov2012}

Conference paper. Developed a fast and accurate technique for temperature
analysis of multiprocessor systems and applied this technique in the context of
reliability optimization. Made extensive use of linear algebra.

\section{Selected Open-Source Projects}

\emph{\link{https://github.com/chain-rule}{Chain Rule (chain-rule on GitHub)}}

Collection of packages used in articles on the
\link[.]{https://blog.ivanukhov.com/}{personal blog about data science}

\emph{\link{https://github.com/learning-on-chip}{Learning on Chip
(learning-on-chip on GitHub)}}

Collection of tools in Rust and Python for processing, simulation, and
prediction of dynamics in a
\link[.]{https://github.com/google/cluster-data}{computer cluster} Used in
research.

\emph{\link{https://github.com/stainless-steel}{Stainless Steel (stainless-steel
on GitHub)}}

Collection of Rust packages of general interest in such areas as linear algebra,
probability theory, statistics, signal processing, and databases.

\emph{\link{https://github.com/ready-steady}{Ready Steady (ready-steady on
GitHub)}}

Collection of Go packages of general interest in such areas as linear algebra,
numerical integration, interpolation, probability theory, and statistics.

\emph{\link{https://github.com/markov-chain}{Markov Chain (markov-chain on
GitHub)}}

Collection of Rust packages for high-level simulation of multiprocessor systems
with an emphasis on their thermal dynamics. Used in research.

\emph{\link{https://github.com/turing-complete}{Turing Complete (turing-complete
on GitHub)}}

Collection of Go packages for high-level simulation of multiprocessor systems
with an emphasis on their thermal dynamics. Used in research.

\section{Personal Information}

Born in 1986. Single. Swedish and Russian. Go to the gym.
\link[.]{https://photography.ivanukhov.com/}{Take pictures}
\link[.]{https://blog.ivanukhov.com/}{Blog about data science} Learn the piano.
Enjoy typography and type design.

Reliable, responsible, organized, meticulous. Have high standards for the
written word and the written code. Work well in a team. Work well individually.

\section{Conclusion}

\begingroup
  \bibliographystyle{IEEEtran}
  \bibliography{IEEEabrv,include/bibliography}
\endgroup

\end{document}
