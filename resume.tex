\documentclass[journal]{IEEEtran}
%
% General
%
\usepackage[english]{babel}
\usepackage[utf8]{inputenc}
\usepackage[T1]{fontenc}

%
% Formatting
%
\usepackage{flushend}
\usepackage[hidelinks]{hyperref}
\usepackage{lettrine}
\usepackage{microtype}
\usepackage{pifont}

\usepackage{xcolor}
\definecolor{liu}{rgb}{0,0.7,0.9}

%
% Shortcuts
%
\newcommand{\linkmarker}{\textsuperscript{$\color{liu}\omega$}}
\newcommand{\link}[3][]{\href{#2}{#3#1\linkmarker}}

\newcommand\footfloat[1]{%
  \begingroup
  \renewcommand\thefootnote{}\footnote{#1}%
  \addtocounter{footnote}{-1}%
  \endgroup
}

\newcommand{\slab}[1]{\label{section:#1}}
\newcommand{\sref}[1]{Section~\ref{section:#1}}

\renewcommand{\date}[1]{{\bf #1} \enspace}
\newcommand{\sep}{\ding{72}~}


\title{Curriculum Vitae of Ivan Ukhov}
\author{
  Ivan Ukhov\\
  \link[,]{mailto:ivan.ukhov@gmail.com}{ivan.ukhov@gmail.com}
  \link[,]{https://blog.ivanukhov.com}{blog.ivanukhov.com}
  \link[,]{https://research.ivanukhov.com}{research.ivanukhov.com}\\
  GitHub: \link[,]{https://github.com/IvanUkhov}{IvanUkhov}
  LinkedIn: \link[,]{https://www.linkedin.com/in/IvanUkhov/}{IvanUkhov}
  Twitter: \link{https://twitter.com/IvanUkhov}{IvanUkhov}
}

\begin{document}

\maketitle

\begin{abstract}
In this work, we summarize Ivan Ukhov's education and career, as well as other
aspects related to his personal and professional work. The motivation is to make
it easier for potential employers to rapidly obtain a sufficiently comprehensive
understanding of the candidate's profile and subsequently decide if and how to
proceed with the talent-acquisition process.
\end{abstract}

\begin{IEEEkeywords}
  Curriculum vitae,
  data science,
  machine learning,
  software engineering,
  statistics,
  talent acquisition.
\end{IEEEkeywords}

\bstctlcite{IEEEexample:BSTcontrol}

\section{Introduction}

\lettrine[findent=0.4em, nindent=0em]{\textbf{T}}{alent acquisition} becomes
increasingly more difficult. There are many candidates, and each one has their
own unique background and their own unique skill set, making it challenging to
navigate. Under these circumstances, it is important to be able to make a
reasonably accurate estimation if the candidate in question would be a good fit
for the job at hand. Ivan Ukhov is one such candidate, and this paper presents
his profile to assist in the aforementioned estimation process. In what follows,
we shall refer to Ivan as the candidate.\footfloat{\linkmarker The omega
subscript indicates the presence of interactive content. It is not accessible in
print form; however, it is considered optional.}

The subsequent sections go over several topics in the order of decreasing
importance. \sref{interests} and \ref{section:skills} describe the candidate's
professional interests and technical skills, respectively, which are arguably
the ones dictating the drive and the ability to follow through. \sref{work},
\ref{section:teaching}, and \ref{section:learning} present his experience in
terms of working, teaching, and learning, respectively. In \sref{education}, the
candidate's formal education is summarized. \sref{publications} and
\ref{section:projects} list some of his scientific publications and open-source
projects, respectively. In \sref{personal}, a glimpse into the candidate's
personal life is given. Lastly, \sref{conclusion} concludes the paper.

\section{Professional Interests} \slab{interests}

We begin with the candidate's professional interests. They are centered on
decision-making based on evidence combined with prior knowledge, answering
business questions via inferential and predictive modeling, optimization of
business processes by means of learning from data, and development of data
products. The underlying disciplines of interest are data science, machine
learning, software engineering, and statistics.

\section{Technical Skills} \slab{skills}

There are a number of methodologies and technologies the candidate has come
across and acquired throughout his education and career: strong knowledge of
Python and R, including their ecosystems; strong knowledge of SQL; strong
knowledge of TensorFlow; strong knowledge of Google Cloud Platform, including
Vertex AI; strong knowledge of shell scripting, Go, and Rust; good knowledge of
Airflow, Docker, and Kubeflow; good knowledge of software testing; familiarity
with C++, JavaScript, HTML, and CSS; good knowledge of continuous integration
and deployment; proficiency in version control with Git; and proficiency in Vim
and LaTeX.

\section{Work Experience} \slab{work}

There have been two employers: LeoVegas and Voi. The corresponding roles are
given in reverse chronological order, which is also the one used in the
remainder of the paper.

\date{February 2022--Present} \emph{Staff Machine Learning Engineer at Voi}:
\sep Established a platform for developing and running machine-learning projects
to be used throughout the company (Google Cloud Platform, GitHub Actions). \sep
Participated in other projects led by the team, including inferential modeling
based on sensor data from IoT units and image classification (TensorFlow) with
object detection (PyTorch) in constrained environments (mobile devices,
Raspberry Pi). \sep Participated in the recruitment of machine-learning
engineers. \sep Mentored colleagues in machine learning and software
engineering.

\date{November 2019--February 2022} \emph{Head of Data Science at LeoVegas}:
\sep Led the team, ensuring clarity and relevance of direction. \sep Held group
and individual meetings. \sep Performed planning and setting of common and
individual objectives.

\date{February 2018--February 2022} \emph{Data Scientist at LeoVegas}: \sep
Developed an experimentation platform for democratization and decentralization
of hypothesis testing (Google Cloud Platform, Bayesian statistics). \sep
Developed a predictive model for identifying individuals who are at risk of
gambling addiction (gradient boosting, recurrent neural networks). \sep
Developed an inferential model and a natural-language-processing model for
analyzing the results of customer surveys (Bayesian statistics, recurrent neural
networks). \sep Developed a data-driven artificial environment for optimizing
promotion campaigns via reinforcement learning (feedforward neural networks).
\sep Developed a platform for data-science projects, including pipelines for
data preparation, frameworks for predictive modeling, and infrastructures for
scheduling and serving predictive models (Google Cloud Platform). \sep
Participated in other projects led by the team, including causal inference in
observational studies, churn prediction, high-value prediction, inference of
spillover in marketing, lifetime-value prediction, marketing mix modeling, and
recommender systems. \sep Participated in the recruitment of data scientists.
\sep Mentored team members in data science, machine learning, and software
engineering; see also \sref{teaching}. \sep Led research projects on the topic
of problem gambling in collaboration with the Karolinska Institute, Stockholm
University, and Nottingham Trent University.

\section{Teaching Experience} \slab{teaching}

In this section, we summarize the candidate's experience in terms of helping
others to grow professionally.

\date{2018--2019} Supervised two Master's students at LeoVegas on the topic of
identifying problem gamblers using recurrent neural networks and on the topic of
managing promotion campaigns using reinforcement learning.

\date{2011--2017} Supervised one Ph.D. student and five Master's students at
Linköping University in machine learning, software engineering, and computer
systems.

\date{2011--2017} Assisted in the following undergraduate courses at Linköping
University: \link[,]{https://studieinfo.liu.se/kurs/tddb84/ht-2018}{Design
Patterns} \link[,]{https://www.ida.liu.se/~TDDD25/}{Distributed Systems}
\link[,]{https://www.ida.liu.se/~TDDI08/}{Embedded Systems Design}
\link[,]{https://www.ida.liu.se/~TDDD04/}{Software Testing} and
\link[.]{https://www.ida.liu.se/~TDTS07/}{System Design and Methodologies}

\section{Learning Experience} \slab{learning}

In this section, we go over several noteworthy courses that the candidate has
passed throughout the years.

\date{2022} Passed the Machine Learning specialization on Coursera (an updated
classic by Andrew Ng) and hosted a ``book'' club for aspiring machine-learning
practitioners at Voi.

\date{2018--2019} Passed the following specializations on Coursera: Deep
Learning, Machine Learning with TensorFlow on Google Cloud Platform, and Data
Engineering.

\date{2011--2017} Passed the following graduate courses at Linköping University:
Advanced Data Models and Databases, Bayesian Learning, Data Mining and
Statistical Learning, Distributed Systems, Gaussian Random Processes, Malliavin
Calculus and Stochastic Integration, Natural Language Processing, Neural
Networks with Applications to Vision and Language, Probability Theory, Real-Time
and Embedded Systems, Stochastic Optimization, and Stochastic Processes.

\section{Academic Qualifications} \slab{education}

In this section, the candidate's formal equation is given, which can be broken
down into two periods: Saint Petersburg, Russian, before 2011 and Linköping,
Sweden, after 2011.

\date{2017} \emph{Doctor of Philosophy in Computer Science}, Embedded Systems
Laboratory, Department of Computer and Information Science, Linköping
University.

\date{2010} \emph{Master of Science in Computer Science} with \textsc{honors},
Department of Information and Control Systems, Peter the Great Saint Petersburg
Polytechnic University.

\date{2010} \emph{Specialist in Business Management} with \textsc{honors},
International Graduate School of Management, Peter the Great Saint Petersburg
Polytechnic University.

\date{2008} \emph{Bachelor of Science in Computer Science} with \textsc{honors},
Department of Information and Control Systems, Peter the Great Saint Petersburg
Polytechnic University.

\section{Selected Scientific Publications} \slab{publications}

In this section, we list several notable scientific publications authored by the
candidate, which were produced during his Ph.D. education; see \sref{education}.

\date{2017} \emph{System-Level Analysis and Design under
Uncertainty}~\cite{ukhov2017d}: In a Ph.D. dissertation, summarized and
unified the individual pieces of research listed below.

\date{2017} \emph{Fine-Grained Long-Range Prediction of Resource Usage in
Computer Clusters}~\cite{ukhov2017b}: Studied resource usage in a
\link[.]{https://github.com/google/cluster-data}{computer cluster} Constructed
an efficient pipeline for data processing and devised a recurrent neural network
for forecasting resource usage. Made use of TensorFlow.

\date{2017} \emph{Fast Synthesis of Power and Temperature Profiles for the
Development of Data-Driven Resource Managers}~\cite{ukhov2017c}: Studied network
traffic in a \link[.]{https://github.com/google/cluster-data}{computer cluster}
Developed an infrastructure for simulating systems processing user requests with
the goal of providing large amounts of synthetic yet realistic data to
facilitate the development of resource managers powered by machine learning.

\date{2017} \emph{Probabilistic Analysis of Electronic Systems via Adaptive
Hierarchical Interpolation}~\cite{ukhov2017a}: Developed an efficient framework
for probabilistic analysis of electronic systems based on adaptive hierarchical
interpolation on sparse grids. Leveraged advanced topics in numerical analysis.

\date{2015} \emph{Temperature-Centric Reliability Analysis and Optimization of
Electronic Systems under Process Variation}~\cite{ukhov2015}: Developed an
efficient probabilistic framework for reliability analysis of electronic systems
using polynomial regression and applied this framework in the context of energy
optimization.

\date{2014} \emph{Probabilistic Analysis of Power and Temperature under Process
Variation for Electronic-System Design}~\cite{ukhov2014b}: Developed an
efficient probabilistic framework for temperature analysis of electronic systems
under process variation based on polynomial regression. Made use of advanced
topics in probability theory and numerical analysis.

\date{2014} \emph{Statistical Analysis of Process Variation Based on Indirect
Measurements for Electronic-System Design}~\cite{ukhov2014a}: Developed an
efficient statistical framework for characterizing variations in parameters of a
technological process based on indirect measurements. Made use of Bayesian
inference.

\date{2012} \emph{Steady-State Dynamic Temperature Analysis and Reliability
Optimization for Embedded Multiprocessor Systems}~\cite{ukhov2012}: Developed a
fast and accurate technique for temperature analysis of multiprocessor systems
under periodic workload and applied this technique in the context of reliability
optimization. Made use of advanced linear algebra.

\section{Selected Open-Source Projects} \slab{projects}

In this section, several of the candidate's open-source projects are given. Each
item below corresponds to an organization on GitHub where the individual
packages can be found.

\emph{\link[:]{https://github.com/chain-rule}{Chain Rule (chain-rule on
GitHub)}} Developed a collection of packages used in the
\link{https://blog.ivanukhov.com/}{personal blog} about data science.

\emph{\link[:]{https://github.com/learning-on-chip}{Learning on Chip
(learning-on-chip on GitHub)}} Developed a collection of tools in Rust and
Python for processing, simulation, and prediction of dynamics in a
\link{https://github.com/google/cluster-data}{computer cluster} for research
purposes; see \sref{publications}.

\emph{\link[:]{https://github.com/stainless-steel}{Stainless Steel
(stainless-steel on GitHub)}} Developed a collection of Rust packages of general
interest in such interrelated areas as linear algebra, probability theory,
statistics, signal processing, and relational databases.

\emph{\link[:]{https://github.com/ready-steady}{Ready Steady (ready-steady on
GitHub)}} Developed a collection of Go packages of general interest in such
interrelated areas as linear algebra, numerical integration, interpolation,
probability theory, and statistics.

\emph{\link[:]{https://github.com/markov-chain}{Markov Chain (markov-chain on
GitHub)}} Developed a collection of Rust packages for high-level simulation of
multiprocessor systems with an emphasis on their thermal dynamics, which was
used in research; see \sref{publications}.

\emph{\link[:]{https://github.com/turing-complete}{Turing Complete
(turing-complete on GitHub)}} Developed a collection of Go packages for
high-level simulation of multiprocessor systems with an emphasis on their
thermal dynamics, which was used in research; see \sref{publications}.

\section{Personal Information} \slab{personal}

The candidate is reliable, responsible, meticulous, and organized. He has high
standards for the written word and the written code. He works well both in a
team and individually.

The candidate was born in 1986 and is both Swedish and Russian. He is single and
lives in Stockholm. He regularly goes to the gym, likes
\link[,]{https://photography.ivanukhov.com/}{photography}
\link[,]{https://blog.ivanukhov.com/}{blogs about data science} and plays the
piano. He enjoys typography and type design.

\section{Conclusion} \slab{conclusion}

In this work, we have summarized Ivan Ukhov's profile to assist potential
employers in estimating the candidate's relevance for their businesses. It is
well understood, however, that the actual causal impact of the present format on
the efficiency and effectiveness of the talent-acquisition process remains
unknown, which we leave for future work.

\begingroup
  \bibliographystyle{IEEEtran}
  \bibliography{IEEEabrv,include/bibliography}
\endgroup

\end{document}
